% Options for packages loaded elsewhere
\PassOptionsToPackage{unicode}{hyperref}
\PassOptionsToPackage{hyphens}{url}
%
\documentclass[
]{article}
\usepackage{amsmath,amssymb}
\usepackage{lmodern}
\usepackage{ifxetex,ifluatex}
\ifnum 0\ifxetex 1\fi\ifluatex 1\fi=0 % if pdftex
  \usepackage[T1]{fontenc}
  \usepackage[utf8]{inputenc}
  \usepackage{textcomp} % provide euro and other symbols
\else % if luatex or xetex
  \usepackage{unicode-math}
  \defaultfontfeatures{Scale=MatchLowercase}
  \defaultfontfeatures[\rmfamily]{Ligatures=TeX,Scale=1}
\fi
% Use upquote if available, for straight quotes in verbatim environments
\IfFileExists{upquote.sty}{\usepackage{upquote}}{}
\IfFileExists{microtype.sty}{% use microtype if available
  \usepackage[]{microtype}
  \UseMicrotypeSet[protrusion]{basicmath} % disable protrusion for tt fonts
}{}
\makeatletter
\@ifundefined{KOMAClassName}{% if non-KOMA class
  \IfFileExists{parskip.sty}{%
    \usepackage{parskip}
  }{% else
    \setlength{\parindent}{0pt}
    \setlength{\parskip}{6pt plus 2pt minus 1pt}}
}{% if KOMA class
  \KOMAoptions{parskip=half}}
\makeatother
\usepackage{xcolor}
\IfFileExists{xurl.sty}{\usepackage{xurl}}{} % add URL line breaks if available
\IfFileExists{bookmark.sty}{\usepackage{bookmark}}{\usepackage{hyperref}}
\hypersetup{
  pdftitle={MovieLens Recommendation System},
  pdfauthor={Richard Jonyo},
  hidelinks,
  pdfcreator={LaTeX via pandoc}}
\urlstyle{same} % disable monospaced font for URLs
\usepackage[margin=1in]{geometry}
\usepackage{color}
\usepackage{fancyvrb}
\newcommand{\VerbBar}{|}
\newcommand{\VERB}{\Verb[commandchars=\\\{\}]}
\DefineVerbatimEnvironment{Highlighting}{Verbatim}{commandchars=\\\{\}}
% Add ',fontsize=\small' for more characters per line
\usepackage{framed}
\definecolor{shadecolor}{RGB}{248,248,248}
\newenvironment{Shaded}{\begin{snugshade}}{\end{snugshade}}
\newcommand{\AlertTok}[1]{\textcolor[rgb]{0.94,0.16,0.16}{#1}}
\newcommand{\AnnotationTok}[1]{\textcolor[rgb]{0.56,0.35,0.01}{\textbf{\textit{#1}}}}
\newcommand{\AttributeTok}[1]{\textcolor[rgb]{0.77,0.63,0.00}{#1}}
\newcommand{\BaseNTok}[1]{\textcolor[rgb]{0.00,0.00,0.81}{#1}}
\newcommand{\BuiltInTok}[1]{#1}
\newcommand{\CharTok}[1]{\textcolor[rgb]{0.31,0.60,0.02}{#1}}
\newcommand{\CommentTok}[1]{\textcolor[rgb]{0.56,0.35,0.01}{\textit{#1}}}
\newcommand{\CommentVarTok}[1]{\textcolor[rgb]{0.56,0.35,0.01}{\textbf{\textit{#1}}}}
\newcommand{\ConstantTok}[1]{\textcolor[rgb]{0.00,0.00,0.00}{#1}}
\newcommand{\ControlFlowTok}[1]{\textcolor[rgb]{0.13,0.29,0.53}{\textbf{#1}}}
\newcommand{\DataTypeTok}[1]{\textcolor[rgb]{0.13,0.29,0.53}{#1}}
\newcommand{\DecValTok}[1]{\textcolor[rgb]{0.00,0.00,0.81}{#1}}
\newcommand{\DocumentationTok}[1]{\textcolor[rgb]{0.56,0.35,0.01}{\textbf{\textit{#1}}}}
\newcommand{\ErrorTok}[1]{\textcolor[rgb]{0.64,0.00,0.00}{\textbf{#1}}}
\newcommand{\ExtensionTok}[1]{#1}
\newcommand{\FloatTok}[1]{\textcolor[rgb]{0.00,0.00,0.81}{#1}}
\newcommand{\FunctionTok}[1]{\textcolor[rgb]{0.00,0.00,0.00}{#1}}
\newcommand{\ImportTok}[1]{#1}
\newcommand{\InformationTok}[1]{\textcolor[rgb]{0.56,0.35,0.01}{\textbf{\textit{#1}}}}
\newcommand{\KeywordTok}[1]{\textcolor[rgb]{0.13,0.29,0.53}{\textbf{#1}}}
\newcommand{\NormalTok}[1]{#1}
\newcommand{\OperatorTok}[1]{\textcolor[rgb]{0.81,0.36,0.00}{\textbf{#1}}}
\newcommand{\OtherTok}[1]{\textcolor[rgb]{0.56,0.35,0.01}{#1}}
\newcommand{\PreprocessorTok}[1]{\textcolor[rgb]{0.56,0.35,0.01}{\textit{#1}}}
\newcommand{\RegionMarkerTok}[1]{#1}
\newcommand{\SpecialCharTok}[1]{\textcolor[rgb]{0.00,0.00,0.00}{#1}}
\newcommand{\SpecialStringTok}[1]{\textcolor[rgb]{0.31,0.60,0.02}{#1}}
\newcommand{\StringTok}[1]{\textcolor[rgb]{0.31,0.60,0.02}{#1}}
\newcommand{\VariableTok}[1]{\textcolor[rgb]{0.00,0.00,0.00}{#1}}
\newcommand{\VerbatimStringTok}[1]{\textcolor[rgb]{0.31,0.60,0.02}{#1}}
\newcommand{\WarningTok}[1]{\textcolor[rgb]{0.56,0.35,0.01}{\textbf{\textit{#1}}}}
\usepackage{longtable,booktabs,array}
\usepackage{calc} % for calculating minipage widths
% Correct order of tables after \paragraph or \subparagraph
\usepackage{etoolbox}
\makeatletter
\patchcmd\longtable{\par}{\if@noskipsec\mbox{}\fi\par}{}{}
\makeatother
% Allow footnotes in longtable head/foot
\IfFileExists{footnotehyper.sty}{\usepackage{footnotehyper}}{\usepackage{footnote}}
\makesavenoteenv{longtable}
\usepackage{graphicx}
\makeatletter
\def\maxwidth{\ifdim\Gin@nat@width>\linewidth\linewidth\else\Gin@nat@width\fi}
\def\maxheight{\ifdim\Gin@nat@height>\textheight\textheight\else\Gin@nat@height\fi}
\makeatother
% Scale images if necessary, so that they will not overflow the page
% margins by default, and it is still possible to overwrite the defaults
% using explicit options in \includegraphics[width, height, ...]{}
\setkeys{Gin}{width=\maxwidth,height=\maxheight,keepaspectratio}
% Set default figure placement to htbp
\makeatletter
\def\fps@figure{htbp}
\makeatother
\setlength{\emergencystretch}{3em} % prevent overfull lines
\providecommand{\tightlist}{%
  \setlength{\itemsep}{0pt}\setlength{\parskip}{0pt}}
\setcounter{secnumdepth}{-\maxdimen} % remove section numbering
\ifluatex
  \usepackage{selnolig}  % disable illegal ligatures
\fi

\title{MovieLens Recommendation System}
\author{Richard Jonyo}
\date{19 February 2022}

\begin{document}
\maketitle

\hypertarget{executive-summary}{%
\section{1. Executive Summary}\label{executive-summary}}

The goal of this project was to come up with a movie recommendation
system using the MovieLens datasets which consists of 10 million movie
ratings. A movie recommendation system is an information filtering
system that attempts to predict the rating or preference a user would
give to a movie. Recommendation systems are an improvement over the
traditional classification models as they can take many classes of input
and provide similarity ranking based on algorithms hence providing the
user with more accurate results.

This project is part of the HarvardX:PH125.9x Data Science: Capstone
course and we use a smaller subset of the
\href{https://grouplens.org/datasets/movielens/10m/}{MovieLens} dataset
which is 10M that contains 10,677 movies by 69,878 users. The dataset
has genres, and the movies are rated from 0.5 to 5 with increments of
0.5. A movie can be categorized under a number of genres.

The dataset is split into two sets: a training set (edx) and a final
hold-out test set (validation) using code provided by the course. The
objective was for the final algorithm to predict ratings with a root
mean square error (RMSE) of less than 0.86490 versus the actual ratings
included in the validation set.

To develop the recommendation model, a 4-step approach was employed
where the initial model assumed the recommendations agreed with the
naive mean of all movie ratings. The second model added the movie bias
to the initial model since some movies seem to be more popular than
others. The third model added users bias which improved the RMSE
slightly. The final model included the regularized movie and user biases
to improve the RMSE.

Exploratory analysis was conducted on the data using R and R Studio, a
language and a software environment for statistical computing. R
Markdown, a simple formatting syntax for authoring HTML, PDF, and MS
Word documents and a component of R Studio was used to compile the
report.

\hypertarget{methods-and-exploratory-analysis}{%
\section{2. Methods and Exploratory
Analysis}\label{methods-and-exploratory-analysis}}

This sections helps us to understand the structure of the Movilens
dataset for us to gain insights that will aid in a better prediction of
movie ratings. It explains the process and techniques used, including
data cleaning, exploration, visualization, insights gained, and the
modeling approach used.

\hypertarget{required-libraries}{%
\subsubsection{Required libraries}\label{required-libraries}}

The project utilized and loaded several CRAN libraries to assist with
our analysis. The libraries were automatically downloaded and installed
during code execution. These included: tidyverse, caret, data.table,
lubridate, and dplyr libraries.

\begin{Shaded}
\begin{Highlighting}[]
\CommentTok{\# Note: this process could take a couple of minutes}
\ControlFlowTok{if}\NormalTok{(}\SpecialCharTok{!}\FunctionTok{require}\NormalTok{(tidyverse)) }\FunctionTok{install.packages}\NormalTok{(}\StringTok{"tidyverse"}\NormalTok{, }\AttributeTok{repos =} \StringTok{"http://cran.us.r{-}project.org"}\NormalTok{)}
\ControlFlowTok{if}\NormalTok{(}\SpecialCharTok{!}\FunctionTok{require}\NormalTok{(caret)) }\FunctionTok{install.packages}\NormalTok{(}\StringTok{"caret"}\NormalTok{, }\AttributeTok{repos =} \StringTok{"http://cran.us.r{-}project.org"}\NormalTok{)}
\ControlFlowTok{if}\NormalTok{(}\SpecialCharTok{!}\FunctionTok{require}\NormalTok{(data.table)) }\FunctionTok{install.packages}\NormalTok{(}\StringTok{"data.table"}\NormalTok{, }\AttributeTok{repos =} \StringTok{"http://cran.us.r{-}project.org"}\NormalTok{)}

\FunctionTok{library}\NormalTok{(tidyverse)}
\FunctionTok{library}\NormalTok{(caret)}
\FunctionTok{library}\NormalTok{(data.table)}
\FunctionTok{library}\NormalTok{(lubridate)}
\FunctionTok{library}\NormalTok{(dplyr)}
\end{Highlighting}
\end{Shaded}

\hypertarget{the-movielens-dataset}{%
\subsubsection{The MovieLens dataset}\label{the-movielens-dataset}}

The Movielens Dataset (25M) contains 25 million ratings and one million
tag applications applied to 62,000 movies by 162,000 users. The dataset
has no demographic information. Each user is represented by an id and no
other information about the users is provided. For this project we use
the Movielens Dataset (10M) which is a subset of the full dataset. Edx
is the name provided to the subset of the MovieLens dataset that
consists of 9,000,055 observations and 6 columns. Each observation is a
rating provided by a user for a movie. The dataset contains ratings
provided by a total of 69,878 unique users for a total of 10,677 unique
movies. Below is the code that was provided by the course so that this
project can build on it.

\begin{Shaded}
\begin{Highlighting}[]
\CommentTok{\# MovieLens 10M dataset:}
\CommentTok{\# https://grouplens.org/datasets/movielens/10m/}
\CommentTok{\# http://files.grouplens.org/datasets/movielens/ml{-}10m.zip}

\NormalTok{dl }\OtherTok{\textless{}{-}} \FunctionTok{tempfile}\NormalTok{()}
\FunctionTok{download.file}\NormalTok{(}\StringTok{"http://files.grouplens.org/datasets/movielens/ml{-}10m.zip"}\NormalTok{, dl)}

\NormalTok{ratings }\OtherTok{\textless{}{-}} \FunctionTok{fread}\NormalTok{(}\AttributeTok{text =} \FunctionTok{gsub}\NormalTok{(}\StringTok{"::"}\NormalTok{, }\StringTok{"}\SpecialCharTok{\textbackslash{}t}\StringTok{"}\NormalTok{, }\FunctionTok{readLines}\NormalTok{(}\FunctionTok{unzip}\NormalTok{(dl, }\StringTok{"ml{-}10M100K/ratings.dat"}\NormalTok{))),}
                 \AttributeTok{col.names =} \FunctionTok{c}\NormalTok{(}\StringTok{"userId"}\NormalTok{, }\StringTok{"movieId"}\NormalTok{, }\StringTok{"rating"}\NormalTok{, }\StringTok{"timestamp"}\NormalTok{))}

\NormalTok{movies }\OtherTok{\textless{}{-}} \FunctionTok{str\_split\_fixed}\NormalTok{(}\FunctionTok{readLines}\NormalTok{(}\FunctionTok{unzip}\NormalTok{(dl, }\StringTok{"ml{-}10M100K/movies.dat"}\NormalTok{)), }\StringTok{"}\SpecialCharTok{\textbackslash{}\textbackslash{}}\StringTok{::"}\NormalTok{, }\DecValTok{3}\NormalTok{)}
\FunctionTok{colnames}\NormalTok{(movies) }\OtherTok{\textless{}{-}} \FunctionTok{c}\NormalTok{(}\StringTok{"movieId"}\NormalTok{, }\StringTok{"title"}\NormalTok{, }\StringTok{"genres"}\NormalTok{)}

\CommentTok{\# We are using R 4.0.4:}
\NormalTok{movies }\OtherTok{\textless{}{-}} \FunctionTok{as.data.frame}\NormalTok{(movies) }\SpecialCharTok{\%\textgreater{}\%} \FunctionTok{mutate}\NormalTok{(}\AttributeTok{movieId =} \FunctionTok{as.numeric}\NormalTok{(movieId),}
                                           \AttributeTok{title =} \FunctionTok{as.character}\NormalTok{(title),}
                                           \AttributeTok{genres =} \FunctionTok{as.character}\NormalTok{(genres))}


\NormalTok{movielens }\OtherTok{\textless{}{-}} \FunctionTok{left\_join}\NormalTok{(ratings, movies, }\AttributeTok{by =} \StringTok{"movieId"}\NormalTok{)}

\CommentTok{\# Our validation set will be 10\% of MovieLens data}
\FunctionTok{set.seed}\NormalTok{(}\DecValTok{1}\NormalTok{, }\AttributeTok{sample.kind=}\StringTok{"Rounding"}\NormalTok{) }\CommentTok{\# if using R 3.5 or earlier, use \textasciigrave{}set.seed(1)\textasciigrave{}}
\NormalTok{test\_index }\OtherTok{\textless{}{-}} \FunctionTok{createDataPartition}\NormalTok{(}\AttributeTok{y =}\NormalTok{ movielens}\SpecialCharTok{$}\NormalTok{rating, }\AttributeTok{times =} \DecValTok{1}\NormalTok{, }\AttributeTok{p =} \FloatTok{0.1}\NormalTok{, }\AttributeTok{list =} \ConstantTok{FALSE}\NormalTok{)}
\NormalTok{edx }\OtherTok{\textless{}{-}}\NormalTok{ movielens[}\SpecialCharTok{{-}}\NormalTok{test\_index,]}
\NormalTok{temp }\OtherTok{\textless{}{-}}\NormalTok{ movielens[test\_index,]}

\CommentTok{\# Make sure userId and movieId in validation set are also in edx set}
\NormalTok{validation }\OtherTok{\textless{}{-}}\NormalTok{ temp }\SpecialCharTok{\%\textgreater{}\%} 
  \FunctionTok{semi\_join}\NormalTok{(edx, }\AttributeTok{by =} \StringTok{"movieId"}\NormalTok{) }\SpecialCharTok{\%\textgreater{}\%}
  \FunctionTok{semi\_join}\NormalTok{(edx, }\AttributeTok{by =} \StringTok{"userId"}\NormalTok{)}

\CommentTok{\# Add rows removed from validation set back into edx set}
\NormalTok{removed }\OtherTok{\textless{}{-}} \FunctionTok{anti\_join}\NormalTok{(temp, validation)}
\NormalTok{edx }\OtherTok{\textless{}{-}} \FunctionTok{rbind}\NormalTok{(edx, removed)}

\FunctionTok{rm}\NormalTok{(dl, ratings, movies, test\_index, temp, movielens, removed)}
\end{Highlighting}
\end{Shaded}

\hypertarget{splitting-the-edx-dataset}{%
\subsubsection{Splitting the edx
dataset}\label{splitting-the-edx-dataset}}

The edx dataset is divided into two sets: training and test datasets.
The test\_dataset is used to evaluate model when building the model. The
training set has 8,100,048 observations and 6 columns. The validation
set which represents 10\% of the 10M Movielens dataset has the same
features, but with a total of 899,993 occurrences and it is used to
evaluate the final model.

\begin{Shaded}
\begin{Highlighting}[]
\CommentTok{\#We divide into two sets: training and test sets}
\CommentTok{\#The test set will be split into two}
\CommentTok{\#test\_dataset: to evaluate and test the model when building the model}
\CommentTok{\#validation: to evaluate the final model}
\NormalTok{test\_dataset }\OtherTok{\textless{}{-}} \FunctionTok{createDataPartition}\NormalTok{(}\AttributeTok{y =}\NormalTok{ edx}\SpecialCharTok{$}\NormalTok{rating, }\AttributeTok{times =} \DecValTok{1}\NormalTok{, }\AttributeTok{p =} \FloatTok{0.1}\NormalTok{, }\AttributeTok{list =} \ConstantTok{FALSE}\NormalTok{)}
\NormalTok{training\_dataset }\OtherTok{\textless{}{-}}\NormalTok{ edx[}\SpecialCharTok{{-}}\NormalTok{test\_dataset,]}
\NormalTok{temp }\OtherTok{\textless{}{-}}\NormalTok{ edx[test\_dataset,]}
\FunctionTok{dim}\NormalTok{(training\_dataset) }\CommentTok{\#training set has 8,100,048 records}
\end{Highlighting}
\end{Shaded}

\begin{verbatim}
## [1] 8100048       6
\end{verbatim}

We ensure that the userId and movieId are in both training and test sets

\begin{Shaded}
\begin{Highlighting}[]
\NormalTok{test\_dataset }\OtherTok{\textless{}{-}}\NormalTok{ temp }\SpecialCharTok{\%\textgreater{}\%}
  \FunctionTok{semi\_join}\NormalTok{(training\_dataset, }\AttributeTok{by =} \StringTok{"movieId"}\NormalTok{) }\SpecialCharTok{\%\textgreater{}\%}
  \FunctionTok{semi\_join}\NormalTok{(training\_dataset, }\AttributeTok{by =} \StringTok{"userId"}\NormalTok{) }

\CommentTok{\# We add the rows removed from or test dataset into training set}
\NormalTok{removed\_rows }\OtherTok{\textless{}{-}} \FunctionTok{anti\_join}\NormalTok{(temp, test\_dataset)}
\NormalTok{training\_dataset }\OtherTok{\textless{}{-}} \FunctionTok{rbind}\NormalTok{(training\_dataset, removed\_rows)}
\end{Highlighting}
\end{Shaded}

\hypertarget{exploratory-analysis}{%
\subsubsection{Exploratory Analysis}\label{exploratory-analysis}}

We need to understand the edx dataset before starting to develop the
models hence an exploratory analysis is significant.

Below is a quick summary (top five records), and the data types of the
dataset:

\begin{Shaded}
\begin{Highlighting}[]
\FunctionTok{head}\NormalTok{(edx, }\DecValTok{5}\NormalTok{) }\CommentTok{\#Preview edx}
\end{Highlighting}
\end{Shaded}

\begin{verbatim}
##    userId movieId rating timestamp                         title
## 1:      1     122      5 838985046              Boomerang (1992)
## 2:      1     185      5 838983525               Net, The (1995)
## 3:      1     292      5 838983421               Outbreak (1995)
## 4:      1     316      5 838983392               Stargate (1994)
## 5:      1     329      5 838983392 Star Trek: Generations (1994)
##                           genres
## 1:                Comedy|Romance
## 2:         Action|Crime|Thriller
## 3:  Action|Drama|Sci-Fi|Thriller
## 4:       Action|Adventure|Sci-Fi
## 5: Action|Adventure|Drama|Sci-Fi
\end{verbatim}

\begin{Shaded}
\begin{Highlighting}[]
\NormalTok{knitr}\SpecialCharTok{::}\FunctionTok{kable}\NormalTok{(}\FunctionTok{summary}\NormalTok{(edx))           }
\end{Highlighting}
\end{Shaded}

\begin{longtable}[]{@{}
  >{\raggedright\arraybackslash}p{(\columnwidth - 12\tabcolsep) * \real{0.03}}
  >{\raggedright\arraybackslash}p{(\columnwidth - 12\tabcolsep) * \real{0.14}}
  >{\raggedright\arraybackslash}p{(\columnwidth - 12\tabcolsep) * \real{0.14}}
  >{\raggedright\arraybackslash}p{(\columnwidth - 12\tabcolsep) * \real{0.14}}
  >{\raggedright\arraybackslash}p{(\columnwidth - 12\tabcolsep) * \real{0.19}}
  >{\raggedright\arraybackslash}p{(\columnwidth - 12\tabcolsep) * \real{0.18}}
  >{\raggedright\arraybackslash}p{(\columnwidth - 12\tabcolsep) * \real{0.18}}@{}}
\toprule
& userId & movieId & rating & timestamp & title & genres \\
\midrule
\endhead
& Min. : 1 & Min. : 1 & Min. :0.500 & Min. :7.897e+08 & Length:9000055 &
Length:9000055 \\
& 1st Qu.:18124 & 1st Qu.: 648 & 1st Qu.:3.000 & 1st Qu.:9.468e+08 &
Class :character & Class :character \\
& Median :35738 & Median : 1834 & Median :4.000 & Median :1.035e+09 &
Mode :character & Mode :character \\
& Mean :35870 & Mean : 4122 & Mean :3.512 & Mean :1.033e+09 & NA & NA \\
& 3rd Qu.:53607 & 3rd Qu.: 3626 & 3rd Qu.:4.000 & 3rd Qu.:1.127e+09 & NA
& NA \\
& Max. :71567 & Max. :65133 & Max. :5.000 & Max. :1.231e+09 & NA & NA \\
\bottomrule
\end{longtable}

The columns in the edx dataset include userId, movieId, rating,
timestamp, title, and genres.

\begin{verbatim}
## Rows: 9,000,055
## Columns: 6
## $ userId    <int> 1, 1, 1, 1, 1, 1, 1, 1, 1, ...
## $ movieId   <dbl> 122, 185, 292, 316, 329, 35...
## $ rating    <dbl> 5, 5, 5, 5, 5, 5, 5, 5, 5, ...
## $ timestamp <int> 838985046, 838983525, 83898...
## $ title     <chr> "Boomerang (1992)", "Net, T...
## $ genres    <chr> "Comedy|Romance", "Action|C...
\end{verbatim}

There are no missing values

\begin{Shaded}
\begin{Highlighting}[]
\FunctionTok{anyNA}\NormalTok{(edx) }\CommentTok{\#check missing values {-} results to FALSE}
\end{Highlighting}
\end{Shaded}

\begin{verbatim}
## [1] FALSE
\end{verbatim}

We have 69,878 users and 10,677 movies in the edx set

\begin{verbatim}
##   n_users n_movies
## 1   69878    10677
\end{verbatim}

\hypertarget{movie-ratings}{%
\subsubsection{Movie ratings}\label{movie-ratings}}

Pulp Fiction (1994), Forrest Gump (1994) and Silence of the Lambs (1991)
are the highest rated in that order with over 30,000 ratings.The average
rating in the edx dataset was 3.51. The minimum rating for the movies
was 0.5 and the maximum rating was 5.

\begin{verbatim}
##  [1] 5.0 3.0 2.0 4.0 4.5 3.5 1.0 1.5 2.5 0.5
\end{verbatim}

The figure below show the top 10 most popular movies.

\begin{Shaded}
\begin{Highlighting}[]
\CommentTok{\#plotting top 10 most popular movies}
\NormalTok{edx }\SpecialCharTok{\%\textgreater{}\%}
  \FunctionTok{group\_by}\NormalTok{(title) }\SpecialCharTok{\%\textgreater{}\%}
  \FunctionTok{summarize}\NormalTok{(}\AttributeTok{count =} \FunctionTok{n}\NormalTok{()) }\SpecialCharTok{\%\textgreater{}\%}
  \FunctionTok{arrange}\NormalTok{(}\SpecialCharTok{{-}}\NormalTok{count) }\SpecialCharTok{\%\textgreater{}\%}
  \FunctionTok{top\_n}\NormalTok{(}\DecValTok{10}\NormalTok{, count) }\SpecialCharTok{\%\textgreater{}\%}
  \FunctionTok{ggplot}\NormalTok{(}\FunctionTok{aes}\NormalTok{(count, }\FunctionTok{reorder}\NormalTok{(title, count))) }\SpecialCharTok{+}
  \FunctionTok{geom\_bar}\NormalTok{(}\AttributeTok{color =} \StringTok{"black"}\NormalTok{, }\AttributeTok{fill =} \StringTok{"brown"}\NormalTok{, }\AttributeTok{stat =} \StringTok{"identity"}\NormalTok{) }\SpecialCharTok{+}
  \FunctionTok{ggtitle}\NormalTok{(}\StringTok{"Top 10 most popular movies"}\NormalTok{)}\SpecialCharTok{+}
  \FunctionTok{theme}\NormalTok{(}\AttributeTok{plot.title =} \FunctionTok{element\_text}\NormalTok{(}\AttributeTok{hjust =} \FloatTok{0.5}\NormalTok{))}\SpecialCharTok{+}
  \FunctionTok{xlab}\NormalTok{(}\StringTok{"Count of ratings"}\NormalTok{) }\SpecialCharTok{+}
  \FunctionTok{ylab}\NormalTok{(}\ConstantTok{NULL}\NormalTok{) }
\end{Highlighting}
\end{Shaded}

\includegraphics{MovieLensProject_files/figure-latex/Movie Ratings-1.pdf}

The figure below shows is the distribution of movie ratings. The average
rating in the edx dataset was 3.512465. The minimum rating for a movie
was 0.5 and the maximum rating was 5.0. Some movies ratings fall below
the average rating, and some are above. We can see that there is a bias
for some movies which makes their ratings deviate from the mean rating.

\begin{Shaded}
\begin{Highlighting}[]
\NormalTok{mean }\OtherTok{=} \FunctionTok{mean}\NormalTok{(edx}\SpecialCharTok{$}\NormalTok{rating)}\CommentTok{\#mean}

\CommentTok{\# We visualize the training set rating distribution}
\NormalTok{edx }\SpecialCharTok{\%\textgreater{}\%} 
\FunctionTok{ggplot}\NormalTok{(}\FunctionTok{aes}\NormalTok{(rating, }\AttributeTok{y =}\NormalTok{ ..prop..)) }\SpecialCharTok{+}
  \FunctionTok{geom\_bar}\NormalTok{(}\AttributeTok{fill =} \StringTok{"brown"}\NormalTok{) }\SpecialCharTok{+}
  \FunctionTok{ggtitle}\NormalTok{(}\StringTok{"Training set rating distribution"}\NormalTok{)}\SpecialCharTok{+}
  \FunctionTok{theme}\NormalTok{(}\AttributeTok{plot.title =} \FunctionTok{element\_text}\NormalTok{(}\AttributeTok{hjust =} \FloatTok{0.5}\NormalTok{))}\SpecialCharTok{+}
  \FunctionTok{scale\_x\_continuous}\NormalTok{(}\AttributeTok{breaks =} \FunctionTok{c}\NormalTok{(}\FloatTok{0.5}\NormalTok{, }\DecValTok{1}\NormalTok{, }\FloatTok{1.5}\NormalTok{, }\DecValTok{2}\NormalTok{, }\FloatTok{2.5}\NormalTok{, }\DecValTok{3}\NormalTok{, }\FloatTok{3.5}\NormalTok{, }\DecValTok{4}\NormalTok{, }\FloatTok{4.5}\NormalTok{, }\DecValTok{5}\NormalTok{))}\SpecialCharTok{+} 
  \FunctionTok{labs}\NormalTok{(}\AttributeTok{x =} \StringTok{"Movie rating"}\NormalTok{, }\AttributeTok{y =} \StringTok{"No. of ratings"}\NormalTok{)}
\end{Highlighting}
\end{Shaded}

\includegraphics{MovieLensProject_files/figure-latex/Plot: Rating Distribution-1.pdf}

\hypertarget{users-rating}{%
\subsubsection{Users rating}\label{users-rating}}

The plot below shows the distribution of the number of ratings by users.
We notice that users do not have a uniform average rating. Users may
have a bias towards particular movies and some may have a tendency to
rate a movie high or less.

\begin{Shaded}
\begin{Highlighting}[]
\CommentTok{\#We plot the no. of ratings by users}
\NormalTok{edx }\SpecialCharTok{\%\textgreater{}\%} 
  \FunctionTok{count}\NormalTok{(userId) }\SpecialCharTok{\%\textgreater{}\%} 
  \FunctionTok{ggplot}\NormalTok{(}\FunctionTok{aes}\NormalTok{(n, }\AttributeTok{fill =} \StringTok{"brown"}\NormalTok{)) }\SpecialCharTok{+} 
  \FunctionTok{geom\_histogram}\NormalTok{( }\AttributeTok{bins=}\DecValTok{30}\NormalTok{, }\AttributeTok{color=}\StringTok{"black"}\NormalTok{, }\AttributeTok{show.legend =} \ConstantTok{FALSE}\NormalTok{) }\SpecialCharTok{+}
  \FunctionTok{scale\_x\_log10}\NormalTok{() }\SpecialCharTok{+}
  \FunctionTok{labs}\NormalTok{(}\AttributeTok{x =} \StringTok{"Users"}\NormalTok{, }\AttributeTok{y =} \StringTok{"Number of ratings"}\NormalTok{)}\SpecialCharTok{+}
  \FunctionTok{ggtitle}\NormalTok{(}\StringTok{"Distribution of number of ratings by users"}\NormalTok{)}\SpecialCharTok{+}
  \FunctionTok{theme}\NormalTok{(}\AttributeTok{plot.title =} \FunctionTok{element\_text}\NormalTok{(}\AttributeTok{hjust =} \FloatTok{0.5}\NormalTok{))}
\end{Highlighting}
\end{Shaded}

\includegraphics{MovieLensProject_files/figure-latex/Plot: Ratings by Users-1.pdf}

We plot mean rating by users and it is evident that the distribution of
the average ratings is right-skewed, meaning that many users tend to
provide good ratings.

\begin{Shaded}
\begin{Highlighting}[]
\CommentTok{\#We plot mean rating by users}
\NormalTok{edx }\SpecialCharTok{\%\textgreater{}\%} \FunctionTok{group\_by}\NormalTok{(userId) }\SpecialCharTok{\%\textgreater{}\%}
  \FunctionTok{summarise}\NormalTok{(}\AttributeTok{mean\_rating =} \FunctionTok{sum}\NormalTok{(rating)}\SpecialCharTok{/}\FunctionTok{n}\NormalTok{()) }\SpecialCharTok{\%\textgreater{}\%}
  \FunctionTok{ggplot}\NormalTok{(}\FunctionTok{aes}\NormalTok{(mean\_rating, }\AttributeTok{fill =} \StringTok{"brown"}\NormalTok{)) }\SpecialCharTok{+}
  \FunctionTok{geom\_histogram}\NormalTok{( }\AttributeTok{bins=}\DecValTok{30}\NormalTok{, }\AttributeTok{color=}\StringTok{"black"}\NormalTok{, }\AttributeTok{show.legend =} \ConstantTok{FALSE}\NormalTok{) }\SpecialCharTok{+}
  \FunctionTok{labs}\NormalTok{(}\AttributeTok{x =} \StringTok{"Average rating"}\NormalTok{, }\AttributeTok{y =} \StringTok{"Number of users"}\NormalTok{)}\SpecialCharTok{+}
  \FunctionTok{ggtitle}\NormalTok{(}\StringTok{"Distribution of average ratings by users"}\NormalTok{)}\SpecialCharTok{+}
  \FunctionTok{theme}\NormalTok{(}\AttributeTok{plot.title =} \FunctionTok{element\_text}\NormalTok{(}\AttributeTok{hjust =} \FloatTok{0.5}\NormalTok{))}
\end{Highlighting}
\end{Shaded}

\includegraphics{MovieLensProject_files/figure-latex/Plot: Mean Rating by Users-1.pdf}

\hypertarget{year-of-release}{%
\subsubsection{Year of release}\label{year-of-release}}

We need to extract the year of release in the edx dataset into a
separate column that will be of a more usable format.

The figure below shows the movie ratings over the years. We can see that
prior to 1990s movies were highly rated than between the years 1930 and
1970. After 1970, the movie ratings decreased gradually.

\begin{Shaded}
\begin{Highlighting}[]
\CommentTok{\#Plot for average rating through the years}
\NormalTok{edx }\SpecialCharTok{\%\textgreater{}\%} \FunctionTok{group\_by}\NormalTok{(release\_year) }\SpecialCharTok{\%\textgreater{}\%}
  \FunctionTok{summarise}\NormalTok{(}\AttributeTok{n =}\FunctionTok{n}\NormalTok{(), }\AttributeTok{avg =} \FunctionTok{mean}\NormalTok{(rating)) }\SpecialCharTok{\%\textgreater{}\%}
  \FunctionTok{ggplot}\NormalTok{(}\FunctionTok{aes}\NormalTok{(release\_year, avg))}\SpecialCharTok{+}
  \FunctionTok{geom\_point}\NormalTok{()}\SpecialCharTok{+}\FunctionTok{geom\_hline}\NormalTok{(}\AttributeTok{yintercept =}\NormalTok{ mean, }\AttributeTok{color =} \StringTok{"red"}\NormalTok{)}\SpecialCharTok{+}\FunctionTok{labs}\NormalTok{(}\AttributeTok{x=}\StringTok{"Release Year"}\NormalTok{,}\AttributeTok{y=}\StringTok{"Average rating"}\NormalTok{)}\SpecialCharTok{+}\FunctionTok{theme}\NormalTok{(}\AttributeTok{axis.text =} \FunctionTok{element\_text}\NormalTok{(}\AttributeTok{size=}\DecValTok{12}\NormalTok{,}\AttributeTok{face =} \StringTok{"bold"}\NormalTok{))}\SpecialCharTok{+}
  \FunctionTok{ggtitle}\NormalTok{(}\StringTok{\textquotesingle{}Average rating through the years\textquotesingle{}}\NormalTok{)}\SpecialCharTok{+}
  \FunctionTok{theme}\NormalTok{(}\AttributeTok{plot.title =} \FunctionTok{element\_text}\NormalTok{(}\AttributeTok{hjust =} \FloatTok{0.5}\NormalTok{))}
\end{Highlighting}
\end{Shaded}

\includegraphics{MovieLensProject_files/figure-latex/Plot - Rating over Years-1.pdf}

\hypertarget{genre}{%
\subsubsection{Genre}\label{genre}}

The top 3 movie genres which were highly reviewed were Drama (733,296),
Comedy (700,889), and Comedy\textbar Romance (365,468)

\begin{Shaded}
\begin{Highlighting}[]
\CommentTok{\#Top 3 movie genres which were highly reviewed}
\NormalTok{top\_genres }\OtherTok{\textless{}{-}}\NormalTok{ edx }\SpecialCharTok{\%\textgreater{}\%} \FunctionTok{group\_by}\NormalTok{(genres) }\SpecialCharTok{\%\textgreater{}\%} 
  \FunctionTok{summarize}\NormalTok{(}\AttributeTok{count =} \FunctionTok{n}\NormalTok{()) }\SpecialCharTok{\%\textgreater{}\%} \FunctionTok{arrange}\NormalTok{(}\FunctionTok{desc}\NormalTok{(count)) }\SpecialCharTok{\%\textgreater{}\%} \FunctionTok{head}\NormalTok{(}\DecValTok{3}\NormalTok{)}
\NormalTok{top\_genres }\SpecialCharTok{\%\textgreater{}\%}\NormalTok{ knitr}\SpecialCharTok{::}\FunctionTok{kable}\NormalTok{()}
\end{Highlighting}
\end{Shaded}

\begin{longtable}[]{@{}lr@{}}
\toprule
genres & count \\
\midrule
\endhead
Drama & 733296 \\
Comedy & 700889 \\
Comedy\textbar Romance & 365468 \\
\bottomrule
\end{longtable}

The figure below shows a summary of the genres by year.

\begin{Shaded}
\begin{Highlighting}[]
\CommentTok{\#We summarize the popular genres by year}
\NormalTok{genres\_by\_year }\OtherTok{\textless{}{-}}\NormalTok{ edx }\SpecialCharTok{\%\textgreater{}\%} 
  \FunctionTok{separate\_rows}\NormalTok{(genres, }\AttributeTok{sep =} \StringTok{"}\SpecialCharTok{\textbackslash{}\textbackslash{}}\StringTok{|"}\NormalTok{) }\SpecialCharTok{\%\textgreater{}\%} 
  \FunctionTok{select}\NormalTok{(movieId, release\_year, genres) }\SpecialCharTok{\%\textgreater{}\%} 
  \FunctionTok{group\_by}\NormalTok{(release\_year, genres) }\SpecialCharTok{\%\textgreater{}\%} 
  \FunctionTok{summarise}\NormalTok{(}\AttributeTok{count =} \FunctionTok{n}\NormalTok{()) }\SpecialCharTok{\%\textgreater{}\%} \FunctionTok{arrange}\NormalTok{(}\FunctionTok{desc}\NormalTok{(release\_year)) }
\end{Highlighting}
\end{Shaded}

Different periods show certain genres being more popular during those
periods. It is for this reason that we will not include genre into our
prediction model. Majority of the ratings were done between the years
1990 and 2010.

\begin{Shaded}
\begin{Highlighting}[]
\FunctionTok{ggplot}\NormalTok{(genres\_by\_year, }\FunctionTok{aes}\NormalTok{(}\AttributeTok{x =}\NormalTok{ release\_year, }\AttributeTok{y =}\NormalTok{ count)) }\SpecialCharTok{+} 
  \FunctionTok{geom\_col}\NormalTok{(}\FunctionTok{aes}\NormalTok{(}\AttributeTok{fill =}\NormalTok{ genres), }\AttributeTok{position =} \StringTok{\textquotesingle{}dodge\textquotesingle{}}\NormalTok{) }\SpecialCharTok{+} 
  \FunctionTok{ylab}\NormalTok{(}\StringTok{\textquotesingle{}No. of movies\textquotesingle{}}\NormalTok{) }\SpecialCharTok{+} 
  \FunctionTok{xlab}\NormalTok{(}\StringTok{\textquotesingle{}Release year\textquotesingle{}}\NormalTok{) }\SpecialCharTok{+} 
  \FunctionTok{ggtitle}\NormalTok{(}\StringTok{\textquotesingle{}Popularity/year by genre\textquotesingle{}}\NormalTok{)}\SpecialCharTok{+}
  \FunctionTok{theme}\NormalTok{(}\AttributeTok{plot.title =} \FunctionTok{element\_text}\NormalTok{(}\AttributeTok{hjust =} \FloatTok{0.5}\NormalTok{))}
\end{Highlighting}
\end{Shaded}

\includegraphics{MovieLensProject_files/figure-latex/Plot: Genre by Year-1.pdf}

\hypertarget{results}{%
\section{3. Results}\label{results}}

This section presents the prediction approach that was employed,
modeling results and discusses the model performance.

\hypertarget{the-prediction-approach}{%
\subsubsection{The Prediction Approach}\label{the-prediction-approach}}

We evaluated the accuracy using the RMSE which measures the difference
between predicted and observed values. Our goal is to reduce the error
below 0.8649.The RMSE function was defined as follows:

\begin{Shaded}
\begin{Highlighting}[]
\CommentTok{\#function to compute RMSE}
\NormalTok{rmse }\OtherTok{\textless{}{-}} \ControlFlowTok{function}\NormalTok{(true\_ratings, predicted\_ratings)\{}
  \FunctionTok{sqrt}\NormalTok{(}\FunctionTok{mean}\NormalTok{((true\_ratings }\SpecialCharTok{{-}}\NormalTok{ predicted\_ratings)}\SpecialCharTok{\^{}}\DecValTok{2}\NormalTok{))}
\NormalTok{\}}
\end{Highlighting}
\end{Shaded}

\hypertarget{first-model-naive-model}{%
\paragraph{First Model (naive model)}\label{first-model-naive-model}}

For our first model we make a prediction using the mean of the movie
ratings. Also known as the naive model which assumes the movies will
have the same rating regardless of genre or user. The mean obtained was
3.512413

\begin{verbatim}
## [1] 3.512509
\end{verbatim}

We obtain our first RMSE = 1.060242. The error is greater than 1 hence
lacks accuracy. Our Second Model will attempt to improve the error.

\begin{Shaded}
\begin{Highlighting}[]
\NormalTok{first\_rmse }\OtherTok{\textless{}{-}} \FunctionTok{rmse}\NormalTok{(training\_dataset}\SpecialCharTok{$}\NormalTok{rating, mean\_movie\_rating)}

\CommentTok{\#Save RMSE result in a dataframe}
\NormalTok{results }\OtherTok{=} \FunctionTok{data\_frame}\NormalTok{(}\AttributeTok{method =} \StringTok{"Model 1: Using the mean (Naive model)"}\NormalTok{, }\AttributeTok{RMSE =}\NormalTok{ first\_rmse)}
\NormalTok{results }\SpecialCharTok{\%\textgreater{}\%}\NormalTok{ knitr}\SpecialCharTok{::}\FunctionTok{kable}\NormalTok{()}
\end{Highlighting}
\end{Shaded}

\begin{longtable}[]{@{}lr@{}}
\toprule
method & RMSE \\
\midrule
\endhead
Model 1: Using the mean (Naive model) & 1.060242 \\
\bottomrule
\end{longtable}

We can confirm that some movies are more popular than others hence
creating a bias. There seems to be is a slight improvement on our second
model where there RMSE = \texttt{second\_rmse}.

\hypertarget{second-model-movie-bias}{%
\paragraph{Second Model (movie bias)}\label{second-model-movie-bias}}

\begin{Shaded}
\begin{Highlighting}[]
\CommentTok{\#We include bias\_1 to represent the mean rating of the movies}
\NormalTok{movie\_bias }\OtherTok{\textless{}{-}}\NormalTok{ training\_dataset }\SpecialCharTok{\%\textgreater{}\%}
  \FunctionTok{group\_by}\NormalTok{(movieId) }\SpecialCharTok{\%\textgreater{}\%}
  \FunctionTok{summarize}\NormalTok{(}\AttributeTok{bias\_1 =} \FunctionTok{mean}\NormalTok{(rating }\SpecialCharTok{{-}}\NormalTok{ mean\_movie\_rating))}

\CommentTok{\#We plot movie bias}
\NormalTok{movie\_bias }\SpecialCharTok{\%\textgreater{}\%} \FunctionTok{ggplot}\NormalTok{(}\FunctionTok{aes}\NormalTok{(bias\_1)) }\SpecialCharTok{+}
  \FunctionTok{geom\_histogram}\NormalTok{(}\AttributeTok{color =} \StringTok{"black"}\NormalTok{, }\AttributeTok{fill =} \StringTok{"brown"}\NormalTok{, }\AttributeTok{bins =} \DecValTok{20}\NormalTok{) }\SpecialCharTok{+}
  \FunctionTok{xlab}\NormalTok{(}\StringTok{"Movie bias"}\NormalTok{) }\SpecialCharTok{+}
  \FunctionTok{ylab}\NormalTok{(}\StringTok{"Count (n)"}\NormalTok{)}\SpecialCharTok{+}
  \FunctionTok{ggtitle}\NormalTok{(}\StringTok{"Movie effect"}\NormalTok{)}\SpecialCharTok{+}
  \FunctionTok{theme}\NormalTok{(}\AttributeTok{plot.title =} \FunctionTok{element\_text}\NormalTok{(}\AttributeTok{hjust =} \FloatTok{0.5}\NormalTok{))}
\end{Highlighting}
\end{Shaded}

\includegraphics{MovieLensProject_files/figure-latex/MOdel 2 - Movie Bias-1.pdf}

\begin{Shaded}
\begin{Highlighting}[]
\CommentTok{\#Testing RMSE by adding the movie bias to our second model}
\CommentTok{\#There is a slight improvement on our second model}
\NormalTok{predictions }\OtherTok{\textless{}{-}}\NormalTok{ mean\_movie\_rating }\SpecialCharTok{+}\NormalTok{ training\_dataset }\SpecialCharTok{\%\textgreater{}\%}
  \FunctionTok{left\_join}\NormalTok{(movie\_bias, }\AttributeTok{by =} \StringTok{"movieId"}\NormalTok{) }\SpecialCharTok{\%\textgreater{}\%}
  \FunctionTok{pull}\NormalTok{(bias\_1)}
\NormalTok{second\_rmse }\OtherTok{\textless{}{-}} \FunctionTok{rmse}\NormalTok{(predictions, test\_dataset}\SpecialCharTok{$}\NormalTok{rating)}
\NormalTok{results }\OtherTok{\textless{}{-}} \FunctionTok{bind\_rows}\NormalTok{(results,}
                          \FunctionTok{data\_frame}\NormalTok{(}\AttributeTok{method=}\StringTok{"Model 2: Mean + movie bias"}\NormalTok{,}
                                     \AttributeTok{RMSE =}\NormalTok{ second\_rmse))}
\NormalTok{results }\SpecialCharTok{\%\textgreater{}\%}\NormalTok{ knitr}\SpecialCharTok{::}\FunctionTok{kable}\NormalTok{()}
\end{Highlighting}
\end{Shaded}

\begin{longtable}[]{@{}lr@{}}
\toprule
method & RMSE \\
\midrule
\endhead
Model 1: Using the mean (Naive model) & 1.060242 \\
Model 2: Mean + movie bias & 1.167344 \\
\bottomrule
\end{longtable}

\hypertarget{third-model-movie-user-biases}{%
\paragraph{Third Model (movie \& user
biases)}\label{third-model-movie-user-biases}}

As seen previously users can add bias by rating some movies very highly
or very low. These extreme ratings adds this bias to our mode. Having
included movie and user biases the error is slightly lower where there
RMSE = 0.86397.

\begin{Shaded}
\begin{Highlighting}[]
\NormalTok{users\_bias }\OtherTok{\textless{}{-}}\NormalTok{ training\_dataset }\SpecialCharTok{\%\textgreater{}\%}
  \FunctionTok{left\_join}\NormalTok{(movie\_bias, }\AttributeTok{by =} \StringTok{"movieId"}\NormalTok{) }\SpecialCharTok{\%\textgreater{}\%}
  \FunctionTok{group\_by}\NormalTok{(userId) }\SpecialCharTok{\%\textgreater{}\%}
  \FunctionTok{summarize}\NormalTok{(}\AttributeTok{bias\_2 =} \FunctionTok{mean}\NormalTok{(rating }\SpecialCharTok{{-}}\NormalTok{ mean\_movie\_rating }\SpecialCharTok{{-}}\NormalTok{ bias\_1))}

\CommentTok{\#We plot movie + User bias}
\NormalTok{users\_bias }\SpecialCharTok{\%\textgreater{}\%} \FunctionTok{ggplot}\NormalTok{(}\FunctionTok{aes}\NormalTok{(bias\_2)) }\SpecialCharTok{+}
  \FunctionTok{geom\_histogram}\NormalTok{(}\AttributeTok{color =} \StringTok{"black"}\NormalTok{, }\AttributeTok{fill =} \StringTok{"brown"}\NormalTok{, }\AttributeTok{bins =} \DecValTok{20}\NormalTok{) }\SpecialCharTok{+}
  \FunctionTok{xlab}\NormalTok{(}\StringTok{"User and movie bias"}\NormalTok{) }\SpecialCharTok{+}
  \FunctionTok{ylab}\NormalTok{(}\StringTok{"Count (n)"}\NormalTok{)}\SpecialCharTok{+}
  \FunctionTok{ggtitle}\NormalTok{(}\StringTok{"Movie + User effect"}\NormalTok{)}\SpecialCharTok{+}
  \FunctionTok{theme}\NormalTok{(}\AttributeTok{plot.title =} \FunctionTok{element\_text}\NormalTok{(}\AttributeTok{hjust =} \FloatTok{0.5}\NormalTok{))}
\end{Highlighting}
\end{Shaded}

\includegraphics{MovieLensProject_files/figure-latex/MOdel 3 - Movie \& User Bias-1.pdf}

\begin{Shaded}
\begin{Highlighting}[]
\NormalTok{predictions }\OtherTok{\textless{}{-}}\NormalTok{ test\_dataset }\SpecialCharTok{\%\textgreater{}\%}
  \FunctionTok{left\_join}\NormalTok{(movie\_bias, }\AttributeTok{by =} \StringTok{"movieId"}\NormalTok{) }\SpecialCharTok{\%\textgreater{}\%}
  \FunctionTok{left\_join}\NormalTok{(users\_bias, }\AttributeTok{by =} \StringTok{"userId"}\NormalTok{) }\SpecialCharTok{\%\textgreater{}\%}
  \FunctionTok{mutate}\NormalTok{(}\AttributeTok{new\_pred =}\NormalTok{ mean\_movie\_rating }\SpecialCharTok{+}\NormalTok{ bias\_1 }\SpecialCharTok{+}\NormalTok{ bias\_2) }\SpecialCharTok{\%\textgreater{}\%}
  \FunctionTok{pull}\NormalTok{(new\_pred)}
\NormalTok{  third\_rmse }\OtherTok{\textless{}{-}} \FunctionTok{RMSE}\NormalTok{(predictions, test\_dataset}\SpecialCharTok{$}\NormalTok{rating)}
\NormalTok{  results }\OtherTok{\textless{}{-}} \FunctionTok{bind\_rows}\NormalTok{(results,}
                       \FunctionTok{data\_frame}\NormalTok{(}\AttributeTok{method=}\StringTok{"Model 3: Mean + movie + user bias"}\NormalTok{,}
                                  \AttributeTok{RMSE =}\NormalTok{ third\_rmse))}
\NormalTok{  results }\SpecialCharTok{\%\textgreater{}\%}\NormalTok{ knitr}\SpecialCharTok{::}\FunctionTok{kable}\NormalTok{()}
\end{Highlighting}
\end{Shaded}

\begin{longtable}[]{@{}lr@{}}
\toprule
method & RMSE \\
\midrule
\endhead
Model 1: Using the mean (Naive model) & 1.0602420 \\
Model 2: Mean + movie bias & 1.1673443 \\
Model 3: Mean + movie + user bias & 0.8659736 \\
\bottomrule
\end{longtable}

\hypertarget{fourth-model-regularized-movie-user-biases}{%
\paragraph{Fourth Model (regularized movie \& user
biases)}\label{fourth-model-regularized-movie-user-biases}}

Some movies are rated by very few users, this can increase RMSE.
Regularization allows for reduced errors caused by movies with few
ratings which can influence the prediction and eventually influence the
error. The final RMSE obtained for our final model was 0.8564223 which
is less than the target RMSE of 0.86490 hence providing a better
accuracy.

\begin{Shaded}
\begin{Highlighting}[]
\NormalTok{  lambdas }\OtherTok{\textless{}{-}} \FunctionTok{seq}\NormalTok{(}\AttributeTok{from=}\DecValTok{0}\NormalTok{, }\AttributeTok{to=}\DecValTok{10}\NormalTok{, }\AttributeTok{by=}\FloatTok{0.25}\NormalTok{)}
\NormalTok{  rmses }\OtherTok{\textless{}{-}} \FunctionTok{sapply}\NormalTok{(lambdas, }\ControlFlowTok{function}\NormalTok{(x)\{}
    
     \CommentTok{\#Adjust mean by movie effect and penalize low number on ratings}
\NormalTok{    movie\_bias }\OtherTok{\textless{}{-}}\NormalTok{ training\_dataset }\SpecialCharTok{\%\textgreater{}\%}
      \FunctionTok{group\_by}\NormalTok{(movieId) }\SpecialCharTok{\%\textgreater{}\%}
      \FunctionTok{summarize}\NormalTok{(}\AttributeTok{movie\_bias =} \FunctionTok{sum}\NormalTok{(rating }\SpecialCharTok{{-}}\NormalTok{ mean\_movie\_rating)}\SpecialCharTok{/}\NormalTok{(}\FunctionTok{n}\NormalTok{()}\SpecialCharTok{+}\NormalTok{x))}
    
    \CommentTok{\#Adjust mean by user + movie effect and penalize low number of ratings}
\NormalTok{    user\_bias }\OtherTok{\textless{}{-}}\NormalTok{ training\_dataset }\SpecialCharTok{\%\textgreater{}\%}
      \FunctionTok{left\_join}\NormalTok{(movie\_bias, }\AttributeTok{by =} \StringTok{"movieId"}\NormalTok{) }\SpecialCharTok{\%\textgreater{}\%}
      \FunctionTok{group\_by}\NormalTok{(userId) }\SpecialCharTok{\%\textgreater{}\%}
      \FunctionTok{summarize}\NormalTok{(}\AttributeTok{user\_bias =} \FunctionTok{sum}\NormalTok{(rating }\SpecialCharTok{{-}}\NormalTok{ movie\_bias }\SpecialCharTok{{-}}\NormalTok{ mean\_movie\_rating)}\SpecialCharTok{/}\NormalTok{(}\FunctionTok{n}\NormalTok{()}\SpecialCharTok{+}\NormalTok{x))}
\NormalTok{    predictions }\OtherTok{\textless{}{-}}\NormalTok{ test\_dataset }\SpecialCharTok{\%\textgreater{}\%}
      \FunctionTok{left\_join}\NormalTok{(movie\_bias, }\AttributeTok{by =} \StringTok{"movieId"}\NormalTok{) }\SpecialCharTok{\%\textgreater{}\%}
      \FunctionTok{left\_join}\NormalTok{(user\_bias, }\AttributeTok{by =} \StringTok{"userId"}\NormalTok{) }\SpecialCharTok{\%\textgreater{}\%}
      \FunctionTok{mutate}\NormalTok{(}\AttributeTok{new\_pred =}\NormalTok{ mean\_movie\_rating }\SpecialCharTok{+}\NormalTok{ movie\_bias }\SpecialCharTok{+}\NormalTok{ user\_bias) }\SpecialCharTok{\%\textgreater{}\%}
      \FunctionTok{pull}\NormalTok{(new\_pred)}
    \FunctionTok{return}\NormalTok{(}\FunctionTok{rmse}\NormalTok{(predictions, test\_dataset}\SpecialCharTok{$}\NormalTok{rating))}
\NormalTok{  \})}
\end{Highlighting}
\end{Shaded}

We plotted RMSE against lambdas to obtain optimal lambda. According to
the plot below lambda at 4.5 produced the lowest RMSE.

\begin{Shaded}
\begin{Highlighting}[]
  \FunctionTok{qplot}\NormalTok{(lambdas, rmses, }\AttributeTok{color =} \FunctionTok{I}\NormalTok{(}\StringTok{"brown"}\NormalTok{)) }
\end{Highlighting}
\end{Shaded}

\includegraphics{MovieLensProject_files/figure-latex/Plot: Lamdas-1.pdf}

\begin{Shaded}
\begin{Highlighting}[]
  \CommentTok{\#We calculate the regularized accuracy with the best lambda}
\NormalTok{  lamd }\OtherTok{\textless{}{-}}\NormalTok{ lambdas[}\FunctionTok{which.min}\NormalTok{(rmses)] }\CommentTok{\#best lamda}
\NormalTok{  lamd}
\end{Highlighting}
\end{Shaded}

\begin{verbatim}
## [1] 4.5
\end{verbatim}

\begin{Shaded}
\begin{Highlighting}[]
\NormalTok{  movie\_bias }\OtherTok{\textless{}{-}}\NormalTok{ edx }\SpecialCharTok{\%\textgreater{}\%} 
    \FunctionTok{group\_by}\NormalTok{(movieId) }\SpecialCharTok{\%\textgreater{}\%}
    \FunctionTok{summarize}\NormalTok{(}\AttributeTok{movie\_bias =} \FunctionTok{sum}\NormalTok{(rating }\SpecialCharTok{{-}}\NormalTok{ mean\_movie\_rating)}\SpecialCharTok{/}\NormalTok{(}\FunctionTok{n}\NormalTok{()}\SpecialCharTok{+}\NormalTok{lamd))}

\NormalTok{  user\_bias }\OtherTok{\textless{}{-}}\NormalTok{ edx }\SpecialCharTok{\%\textgreater{}\%} 
    \FunctionTok{left\_join}\NormalTok{(movie\_bias, }\AttributeTok{by=}\StringTok{"movieId"}\NormalTok{) }\SpecialCharTok{\%\textgreater{}\%}
    \FunctionTok{group\_by}\NormalTok{(userId) }\SpecialCharTok{\%\textgreater{}\%}
    \FunctionTok{summarize}\NormalTok{(}\AttributeTok{user\_bias =} \FunctionTok{sum}\NormalTok{(rating }\SpecialCharTok{{-}}\NormalTok{ movie\_bias }\SpecialCharTok{{-}}\NormalTok{ mean\_movie\_rating)}\SpecialCharTok{/}\NormalTok{(}\FunctionTok{n}\NormalTok{()}\SpecialCharTok{+}\NormalTok{lamd))}
  
\NormalTok{  predictions }\OtherTok{\textless{}{-}}\NormalTok{ test\_dataset }\SpecialCharTok{\%\textgreater{}\%} 
    \FunctionTok{left\_join}\NormalTok{(movie\_bias, }\AttributeTok{by =} \StringTok{"movieId"}\NormalTok{) }\SpecialCharTok{\%\textgreater{}\%}
    \FunctionTok{left\_join}\NormalTok{(user\_bias, }\AttributeTok{by =} \StringTok{"userId"}\NormalTok{) }\SpecialCharTok{\%\textgreater{}\%}
    \FunctionTok{mutate}\NormalTok{(}\AttributeTok{new\_pred =}\NormalTok{ mean\_movie\_rating }\SpecialCharTok{+}\NormalTok{ movie\_bias }\SpecialCharTok{+}\NormalTok{ user\_bias) }\SpecialCharTok{\%\textgreater{}\%}
    \FunctionTok{pull}\NormalTok{(new\_pred)}
\NormalTok{  fourth\_rmse }\OtherTok{\textless{}{-}} \FunctionTok{rmse}\NormalTok{(predictions, test\_dataset}\SpecialCharTok{$}\NormalTok{rating)}
\NormalTok{  results }\OtherTok{\textless{}{-}} \FunctionTok{bind\_rows}\NormalTok{(results,}
                       \FunctionTok{data\_frame}\NormalTok{(}\AttributeTok{method=}\StringTok{"Model 4: Mean + movie + user bias + Regularisation"}\NormalTok{,}
                                  \AttributeTok{RMSE =}\NormalTok{ fourth\_rmse))}
\NormalTok{  results }\SpecialCharTok{\%\textgreater{}\%}\NormalTok{ knitr}\SpecialCharTok{::}\FunctionTok{kable}\NormalTok{()}
\end{Highlighting}
\end{Shaded}

\begin{longtable}[]{@{}lr@{}}
\toprule
method & RMSE \\
\midrule
\endhead
Model 1: Using the mean (Naive model) & 1.0602420 \\
Model 2: Mean + movie bias & 1.1673443 \\
Model 3: Mean + movie + user bias & 0.8659736 \\
Model 4: Mean + movie + user bias + Regularisation & 0.8574155 \\
\bottomrule
\end{longtable}

\hypertarget{conclusion}{%
\section{4. Conclusion}\label{conclusion}}

We employed a 4-step approach to come up with a movie recommendation
model. The final model (fourth model) is the preferred movie
recommendation model since its RMSE is below the target of 0.8649. The
limitations of this project included having movies that were rated by a
small number of users which were not credible enough. Similarly, users
who rated only a small number of movies should not be taken into account
to remove the biases. Preferably we should only consider users who have
given at least 50 ratings. It is also suggested that other data such as
age, user behavior, gender, etc. could further enhance the final model.

\end{document}
